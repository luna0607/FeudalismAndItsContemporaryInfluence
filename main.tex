%% LyX 2.3.1 created this file.  For more info, see http://www.lyx.org/.
%% Do not edit unless you really know what you are doing.
\documentclass[oneside]{ctexbook}
\usepackage{fontspec}
\setcounter{secnumdepth}{3}
\setcounter{tocdepth}{3}
\begin{document}
\title{封建主义及其当代影响}
\author{姬金铎}
\maketitle

\chapter{马克思关于反对封建主义与人的解放的理论解析}

马克思关于人的解放的理论包含着十分丰富的思想内容.然而在对这个问题的研究中,多数的研究者大都是从哲学与马克思思想形成和发展的角度去进行研究的.而真正从这个问题的政治的角度进行研究的成果还不多见.而从政治的角度来研究和理解这个问题对我们理解马克思政治思想,对指导我国的社会主义建设和改革开放的伟大事业,都有着极为重要的意义.下面我想就这个问题谈一点自己的看法.

\section{马克思从人的解放的高度对封建主义进行了批判}

实现人的解放,是马克思一生关注的一个核心问题,也是马克思毕生研究的一个重要课题.从马克思的年轻时代起,人的解放问题就成了他关注的焦点.马克思对人的解放的研究是从人的被束缚的事实开始的.那么,是谁束缚了人的发展而使他们处于没有得到解放的受奴役的地位呢?马克思的回答是封建主义和封建专制制度.封建专制制度究竟从哪些方面束缚着人,使人得不到解放呢?对此马克思在他的早期著作中做过比较透彻的分析.

第一,马克思认为,从本质上来说,封建专制制度是一种轻视人,奴役人的制度.马克思说:``专制制度的唯一原则就是轻视人类,使人不成其为人,\ldots 专制君主总是把人看的很下贱.''马克思还讲到:``君主政体的原则总的来说就是轻视人,蔑视人,使人不成其为人\ldots 哪里君主制的原则占优势,哪里的人就占少数;哪里君主制的原则是天经地义的,哪里就根本没有人了.''马克思认为,正是因为封建专制制度的这种轻视人,奴役人的性质,就是这种制度成为具有兽性的制度.他写道,``专制制度必然具有兽性,并且和人性是不相容的.兽的关系只能靠兽性来维持.''马克思还认为,封建专制制度是把人变成工具的制度,而不是使人成为目的的制度.

第二,马克思认为,封建专制制度这种轻视人,蔑视人的性质,使人具有依附性.马克思说:那些不感到自己是人的人,就像繁殖出来的奴隶和马匹一样,完全成为了主人的附属品.世袭的评价就是这个社会的一切.这个世界是属于他们的.他们认为这个世界就是它现在这个样子的
\end{document}
